\documentclass[11pt]{article} % for final 
\usepackage[top=0.8in, bottom=1.2in, left=1in, right=1in]{geometry}
% The amssymb package provides various useful mathematical symbols
\usepackage{amsmath}
\usepackage{amsfonts}
\usepackage{amssymb}
\usepackage{graphicx}
\usepackage{xspace}
\usepackage{bm}
\usepackage{threeparttable}
\usepackage{subfigure}
\usepackage{afterpage}
\usepackage{soul}
\usepackage{listings}
\usepackage[usenames,dvipsnames]{xcolor}
\usepackage{hyperref}

\hypersetup{
  colorlinks=true,
  linkcolor=blue,
  citecolor=blue,
  urlcolor=blue
}
  
%%
%% Julia definition (c) 2014 Jubobs
%%
\lstdefinelanguage{Julia}%
  {morekeywords={abstract,begin,break,case,catch,const,continue,do,else,elseif,%
      end,export,false,for,function,struct,import,importall,if,in,%
      macro,module,otherwise,quote,return,switch,true,try,type,typealias,%
      using,while},%
    sensitive=true,%
    alsoother={\$},%
    morecomment=[l]\#,%
    morecomment=[n]{\#=}{=\#},%
    morestring=[s]{``}{``},%
    morestring=[m]{'}{'},%
  }[keywords,comments,strings]%
  
  \lstset{%
    language         = Julia,
    basicstyle       = \ttfamily,
    keywordstyle     = \small\bfseries\color{blue},
    stringstyle      = \color{magenta},
    commentstyle     = \color{ForestGreen},
    showstringspaces = false,
}



\newcommand{\add}[1]{\textcolor{red}{#1}}
\newcommand{\fnc}[1]{\ensuremath{\mathcal{#1}}}
%\newcommand{\vfc}[1]{\ensuremath{\boldsymbol{\mathcal{#1}}}}
\newcommand{\vfc}[1]{\ensuremath{\mathcal{#1}}}
\newcommand{\mat}[1]{\ensuremath{\mathsf{#1}}}
\newcommand{\M}[0]{\mat{H}}
\newcommand{\Dx}[0]{\mat{D}_{x}}
\newcommand{\Dy}[0]{\mat{D}_{y}}
\newcommand{\Dz}[0]{\mat{D}_{z}}

\newcommand{\Q}[1]{\mat{Q}_{#1}}
\newcommand{\Qx}[0]{\mat{Q}_{x}}
\newcommand{\Qy}[0]{\mat{Q}_{y}}
\newcommand{\Qz}[0]{\mat{Q}_{z}}
\newcommand{\QAx}[0]{\mat{S}_{x}}
\newcommand{\QAy}[0]{\mat{S}_{y}}
\newcommand{\QAz}[0]{\mat{S}_{z}}
\newcommand{\Ex}[0]{\mat{E}_{x}}
\newcommand{\Ey}[0]{\mat{E}_{y}}
\newcommand{\Ez}[0]{\mat{E}_{z}}
\newcommand{\B}[0]{\mat{B}}
\newcommand{\R}[0]{\mat{R}}

\newcommand{\ignore}[1]{}
\newcommand{\etal}[0]{{\em et~al.\@}\xspace}
\newcommand{\eg}[0]{{e.g.\@}\xspace}
\newcommand{\ie}[0]{{i.e.\@}\xspace}
\newcommand{\viz}[0]{{viz.\@}\xspace}
\newcommand{\resp}[0]{{resp.\@}\xspace}

\title{Using SummationByParts.jl to Discretize the Euler Equations with a
  discontinuous basis}

\author{Jason~E. Hicken}

\begin{document}

\maketitle

\section*{The ``Math''}

Consider the weak form of the Euler equations on a single element with domain
$\Omega$:
\begin{equation*}
  - \int_{\Omega}  \frac{\partial \vfc{V}^{T}}{\partial x_{i}} 
  \vfc{F}_{i}(\vfc{U}) \; d\Omega
  + \sum_{\nu} \int_{\Gamma_{\nu}} \vfc{V}^{T} \hat{\vfc{F}}(\vfc{U},\vfc{U}_{\nu},n)  \; d\Gamma
  = 0,\qquad \forall\; \vfc{V} \in P_{d}^{\text{nvar}}(\Omega),
\end{equation*}
where $\bigcup_{\nu} \Gamma_{\nu}$ denotes the boundary of $\Omega$ and
$P_{d}^{\text{nvar}}(\Omega)$ is the space of nvar-vector-valued polynomials of
total degree $d$ on $\Omega$.

The Euler flux in the Cartesian direction $i$ is given by $\vfc{F}_{i}$, while
the function $\hat{\vfc{F}}(\vfc{U}, \vfc{U}_{\nu}, n)$ represents a numerical
flux function.  The numerical flux depends on the discrete solution, $\vfc{U}$,
the ``boundary'' data, $\vfc{U}_{\nu}$, and the normal vector, $n$.  The
boundary data may come from the boundary conditions or the adjacent element,
depending on whether $\Gamma_{\nu}$ is a boundary of the global domain or an
interface.

After transforming to the standard reference domain, the SBP discretization of
the Euler equations can be written as follows:
\begin{equation*}
  -\bm{v}^{T} \Q{i}^{T} \bm{f}_{i}(\bm{u}) + \sum_{\nu} \bm{v}^{T} \R_{\nu}^{T} \B_{\nu}
  \hat{\bm{f}}(\R_{\nu} \bm{u}, \bm{u}_{\nu},n_{\nu}) = \bm{0},
  \qquad \forall\; \bm{v} \in \mathbb{R}^{(\text{nvar})n},
\end{equation*}
where $\bm{u}$ is the discrete solution at the cubature nodes of the element
$\Omega$ and $\bm{f}_{i}(\bm{u})$ is the Euler flux in the $\xi_{i}$
computational-space direction.  In addition, the SBP operator $\Q{i} =
\Q{\xi_{i}} \otimes \mat{I}$, where $\mat{I}$ is the $\text{nvar}\times
\text{nvar}$ identity matrix ($4\times 4$ identity in 2D and the $5\times 5$
identity in 3D).

For the discretized boundary integrals, I have introduced 
\begin{itemize}
\item $\bm{u}_{\nu}$, which is the boundary-condition values at the cubature
  nodes of $\Gamma_{\nu}$, or the solution on the adjacent element interpolated
  to the cubature nodes of $\Gamma_{\nu}$;
\item $\R_{\nu}$, which interpolates the solution $\bm{u}$ from the volume nodes
  to the cubature nodes of $\Gamma_{\nu}$;
\item $\B_{\nu}$, which denotes the cubature weights for the face $\Gamma_{\nu}$, and;
\item $\hat{\bm{f}}$, which denotes the numerical flux function.
\end{itemize}

\newpage

\section*{The ``Code''}

I will now describe the steps and appropriate \texttt{SummationByParts.jl}
methods necessary to compute the individual terms.

\subsection*{$-\bm{v}^{T} \Q{i}^{T} \bm{f}_{i}(\bm{u})$}

Assuming that the scaled Jacobian terms, $\frac{1}{J}\nabla_x \boldsymbol{\xi}$,
are precomputed, evaluate the Euler fluxes $\bm{f}_{i}(\bm{u})$ and store in
\texttt{flux[:,:,:,i]}.  Then call
\begin{lstlisting}
  for i = 1:dim
    weakdifferentiate!(sbp, i, view(flux,:,:,:,i), res, trans=true)
  end
  scale!(res, -1.0)
\end{lstlisting}
where \texttt{sbp} is an appropriate SBP operator, and \texttt{res[:,:,:]} is
the residual.

\subsection*{$\R_{\nu} \bm{u}$ and $n_{\nu}$}

Consider interfaces first; I discuss boundary faces below.  To evaluate
$\R_{\nu} \bm{u}$, that is, to interpolate to the cubature nodes of the
$\Gamma_{\nu}$, use
\begin{lstlisting}
  uface = zeros(Tsol, (nvar, sbpface.numnodes, 2, size(ifaces,1)))
  interiorfaceinterpolate!(sbpface, ifaces, u, uface)
\end{lstlisting}
where \texttt{u} is the solution (it has a size \texttt{(nvar, sbp.numnodes,
  numelem)}), \texttt{ifaces} is an array of \texttt{Interfaces}, and
\texttt{uface} is the interpolated solution.  Note that the \texttt{2} in the
initialization of \texttt{uface} is for the states on either side of the
interface.

The object \texttt{sbpface} is an \texttt{SBP.AbstractFace} type that contains
data to perform interpolation and integration on faces.  For the faces of
triangles, this object can be created using the constructor
\begin{lstlisting}
  sbpface = TriFace{Float64}(degree, volcub, refvertices)
\end{lstlisting}
where \texttt{degree} is the degree of the SBP operator, \texttt{volcub} is the
cubature associated with an SBP operator, and \texttt{refvertices} are the
locations of the reference vertices.  You only need one \texttt{sbpface} for
each SBP operator type.

\noindent\textbf{WARNING:} the constructor for TriFace is likely to change.

We also need the scaled normal vector to compute the numerical flux.  This
should probably be precomputed by interpolating $\partial x_{i}/\partial
\xi_{j}$ to the face nodes, computing $\frac{1}{J} \nabla_{x} \boldsymbol{\xi}$,
and then contracting with $n_\xi$.

\subsection*{$\hat{\bm{f}}(\R_{\nu} \bm{u}, \bm{u}_{\nu},n_{\nu})$}

The computation of the numerical flux requires that we loop over the face nodes
in such a way that we visit the ``left'' and ``right'' nodes in a consistent
manner, \ie we are at the same spatial location.  This can be done using
something like the following loop:
\begin{lstlisting}
  fluxface = zeros((nvar, sbpface.numnodes, size(ifaces,1)))
  @inbounds begin
    for (findex, face) in enumerate(ifaces)
      for i = 1:sbpface.numnodes
        iR = sbpface.nbrperm[i,face.orient]
        fluxface[i,findex] =
          fluxfunc(uface[:,i,1,findex], uface[:,iR,2,findex],
                   nrm[:,i,findex])
      end
    end
  end
\end{lstlisting}
The array \texttt{sbpface.nbrperm} maps the face node corresponding to the ``left''
element to the face node corresponding to the ``right'' element.

\noindent\textbf{Remark:} It might make sense to pass \texttt{SummationByParts} a function and let it take care of this loop.

\subsection*{$\bm{v}^{T} \R_{\nu}^{T} \B_{\nu} \hat{\bm{f}}$}

The last step is to scale the fluxes by the cubature weights and then transfer
the face fluxes back to the volume nodes using $\R_{\nu}^{T}$.  These two steps are performed together using
\begin{lstlisting}
  interiorfaceintegrate!(sbpface, ifaces, fluxface, res)
\end{lstlisting}

\subsection*{Boundary Faces}

The process for boundary faces is similar, with some minor differences.

First, the boundary face values are not double valued, so the interpolation
looks like
\begin{lstlisting}
  uface = zeros(Tsol, (nvar, sbpface.numnodes, size(bfaces,1)))
  boundaryfaceinterpolate!(sbpface, bfaces, u, uface)
\end{lstlisting}

Next, there is no need to worry about matching interface nodes, so the flux
loop simplifies:
\begin{lstlisting}
  fluxface = zeros((nvar, sbpface.numnodes, size(bfaces,1)))
  @inbounds begin
    for (findex, face) in enumerate(bfaces)
      for i = 1:sbpface.numnodes
        fluxface[i,findex] =
          bndryfunc(uface[:,i,findex], nrm[:,i,findex])
      end
    end
  end
\end{lstlisting}

\noindent\textbf{Remark:} Depending on the boundary condition, we may need the
spatial location of the cubature node to be included with the call to
\texttt{bndryflux}, or some constant flow state, \eg the far-field conditions.

The final step is mostly unchanged in syntax from the interface case:
\begin{lstlisting}
  boundaryintegrate!(sbpface, bfaces, fluxface, res)
\end{lstlisting}


\end{document}
